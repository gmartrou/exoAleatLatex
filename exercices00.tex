\documentclass[10pt, a4paper]{article}
\usepackage[utf8x]{inputenc}
\usepackage[french]{babel}
\usepackage[T1]{fontenc}
\usepackage{adjustbox}
\newcounter{nexo}           % déclaration du numéro d'exo
\setcounter{nexo}{0}        % initialisation du numero
\newcommand{\exo}[1]{
  \refstepcounter{nexo} 
  \par{{\section*{Exercice \arabic{nexo} : #1}}}\noindent}
\newcommand{\cor}[1]{
  \refstepcounter{nexo}
  \par{{\section*{Correction \arabic{nexo} : #1}}}\noindent}
\newcommand{\vv}[1]{\overrightarrow{#1}}
\newcommand{\coord}[3]{#1\begin{pmatrix}
  #2 \\
  #3
  \end{pmatrix}}
\newcommand{\coordv}[3]{\vv{#1}\begin{pmatrix}
  #2 \\
  #3
  \end{pmatrix}}
\usepackage{graphicx}
\usepackage{eurosym}
\usepackage{textcomp}
\usepackage{amsmath}
\usepackage{geometry}
\geometry{tmargin=0.6cm,lmargin=1cm,rmargin=1cm,bmargin=1.5cm}
\usepackage{xcolor}
\usepackage{dsfont}
\newcommand{\R}{\mathds {R}}
\usepackage{tikz,tkz-tab}
\usetikzlibrary{shapes.misc}
\tikzset{cross/.style={cross out, draw=red, minimum size=2*(#1-\pgflinewidth), inner sep=0pt, outer sep=0pt},
%default radius will be 1pt. 
cross/.default={1pt}}
\usepackage{pgfplots}
\pgfplotsset{compat=1.16}
\usepackage{interval}
\setlength\parindent{0pt}
\usepackage{enumitem}
%---- Style de l'entête pour le grade 10 LLG Paris-Abu Dhabi-----
%\setlength{\parindent}{0.5cm}
\newcommand{\enteteLSMI}[2]
{
  % Lieu - Année
  \noindent{\underline{\LARGE \textbf{Lycée Stendhal}} \hfill \large \textbf{Mathématiques, Classe de 4$^{\text{ème}}$D}


  %  Module - Type de feuille
  \noindent{\ \ \LARGE AEFE - Milan  \large \hfill {#1}}\\
  \noindent{\phantom{Name : \dotfill}}\hfill }  
 \vspace*{-0.3cm}
  \hrule
  \begin{center} \textbf{\textsf{\Large #2 }} \end{center}\vspace*{-0.12cm}
  \hrule
  \vspace*{0.25cm}
}%
\begin{document}%

\enteteLSMI{\today}{Interrogation n°10 : tableaux de signes : sujet A}%

\exo{Vecteurs et coordonnées.}%
Dans le plan muni d'un repère $\left( {{\mathrm{O}};\vec{\imath}, 
\vec{\jmath}} \right)$, on considère les points $A\left(-3; -5\right)$, $B\left(-2; 5
\right)$ et $C\left(2; 2\right)$.\\%
Calculez les coordonnées du point D tel que :%
\begin{center}%
$\vv{BD} = -3 \vv{AB}  +  5 \vv{CA}$%
\end{center}%

\exo{Vecteurs et coordonnées.}%
Dans le plan muni d'un repère $\left( {{\mathrm{O}};\vec{\imath}, 
\vec{\jmath}} \right)$, on considère les points $A\left(-2; -5\right)$, $B\left(-3; 4
\right)$ et $C\left(2; 4\right)$.\\%
Calculez les coordonnées du point D tel que :%
\begin{center}%
$\vv{BD} = -5 \vv{AB}  -  4 \vv{CA}$%
\end{center}%

\exo{Vecteurs et coordonnées.}%
Dans le plan muni d'un repère $\left( {{\mathrm{O}};\vec{\imath}, 
\vec{\jmath}} \right)$, on considère les points $A\left(-1; -2\right)$, $B\left(-4; 1
\right)$ et $C\left(1; 2\right)$.\\%
Calculez les coordonnées du point D tel que :%
\begin{center}%
$\vv{BD} = 4 \vv{AB}  -  2 \vv{CA}$%
\end{center}%

\exo{Vecteurs et coordonnées.}%
Dans le plan muni d'un repère $\left( {{\mathrm{O}};\vec{\imath}, 
\vec{\jmath}} \right)$, on considère les points $A\left(-4; -3\right)$, $B\left(-3; 5
\right)$ et $C\left(3; 2\right)$.\\%
Calculez les coordonnées du point D tel que :%
\begin{center}%
$\vv{BD} = -3 \vv{AB}  +  2 \vv{CA}$%
\end{center}%

\exo{Vecteurs et coordonnées.}%
Dans le plan muni d'un repère $\left( {{\mathrm{O}};\vec{\imath}, 
\vec{\jmath}} \right)$, on considère les points $A\left(-5; -3\right)$, $B\left(-3; 5
\right)$ et $C\left(2; 2\right)$.\\%
Calculez les coordonnées du point D tel que :%
\begin{center}%
$\vv{BD} = -4 \vv{AB}  -  2 \vv{CA}$%
\end{center}%

\exo{Vecteurs et coordonnées.}%
Dans le plan muni d'un repère $\left( {{\mathrm{O}};\vec{\imath}, 
\vec{\jmath}} \right)$, on considère les points $A\left(-1; -5\right)$, $B\left(-2; 2
\right)$ et $C\left(4; 4\right)$.\\%
Calculez les coordonnées du point D tel que :%
\begin{center}%
$\vv{BD} = 4 \vv{AB}  +  5 \vv{CA}$%
\end{center}%

\exo{Vecteurs et coordonnées.}%
Dans le plan muni d'un repère $\left( {{\mathrm{O}};\vec{\imath}, 
\vec{\jmath}} \right)$, on considère les points $A\left(-2; -5\right)$, $B\left(-3; 4
\right)$ et $C\left(1; 3\right)$.\\%
Calculez les coordonnées du point D tel que :%
\begin{center}%
$\vv{BD} = 5 \vv{AB}  +  4 \vv{CA}$%
\end{center}%

\exo{Vecteurs et coordonnées.}%
Dans le plan muni d'un repère $\left( {{\mathrm{O}};\vec{\imath}, 
\vec{\jmath}} \right)$, on considère les points $A\left(-1; -5\right)$, $B\left(-4; 3
\right)$ et $C\left(5; 3\right)$.\\%
Calculez les coordonnées du point D tel que :%
\begin{center}%
$\vv{BD} = 2 \vv{AB}  -  3 \vv{CA}$%
\end{center}%

\exo{Vecteurs et coordonnées.}%
Dans le plan muni d'un repère $\left( {{\mathrm{O}};\vec{\imath}, 
\vec{\jmath}} \right)$, on considère les points $A\left(-1; -4\right)$, $B\left(-1; 3
\right)$ et $C\left(2; 5\right)$.\\%
Calculez les coordonnées du point D tel que :%
\begin{center}%
$\vv{BD} = -3 \vv{AB}  -  5 \vv{CA}$%
\end{center}%

\exo{Vecteurs et coordonnées.}%
Dans le plan muni d'un repère $\left( {{\mathrm{O}};\vec{\imath}, 
\vec{\jmath}} \right)$, on considère les points $A\left(-5; -5\right)$, $B\left(-2; 4
\right)$ et $C\left(1; 1\right)$.\\%
Calculez les coordonnées du point D tel que :%
\begin{center}%
$\vv{BD} = -4 \vv{AB}  +  5 \vv{CA}$%
\end{center}%

\exo{Vecteurs et coordonnées.}%
Dans le plan muni d'un repère $\left( {{\mathrm{O}};\vec{\imath}, 
\vec{\jmath}} \right)$, on considère les points $A\left(-1; -5\right)$, $B\left(-5; 4
\right)$ et $C\left(4; 4\right)$.\\%
Calculez les coordonnées du point D tel que :%
\begin{center}%
$\vv{BD} = 3 \vv{AB}  +  2 \vv{CA}$%
\end{center}%

\exo{Vecteurs et coordonnées.}%
Dans le plan muni d'un repère $\left( {{\mathrm{O}};\vec{\imath}, 
\vec{\jmath}} \right)$, on considère les points $A\left(-1; -2\right)$, $B\left(-5; 2
\right)$ et $C\left(5; 1\right)$.\\%
Calculez les coordonnées du point D tel que :%
\begin{center}%
$\vv{BD} = -3 \vv{AB}  +  5 \vv{CA}$%
\end{center}%

\exo{Vecteurs et coordonnées.}%
Dans le plan muni d'un repère $\left( {{\mathrm{O}};\vec{\imath}, 
\vec{\jmath}} \right)$, on considère les points $A\left(-4; -1\right)$, $B\left(-2; 3
\right)$ et $C\left(2; 1\right)$.\\%
Calculez les coordonnées du point D tel que :%
\begin{center}%
$\vv{BD} = 2 \vv{AB}  +  3 \vv{CA}$%
\end{center}%

\exo{Vecteurs et coordonnées.}%
Dans le plan muni d'un repère $\left( {{\mathrm{O}};\vec{\imath}, 
\vec{\jmath}} \right)$, on considère les points $A\left(-4; -5\right)$, $B\left(-2; 2
\right)$ et $C\left(1; 1\right)$.\\%
Calculez les coordonnées du point D tel que :%
\begin{center}%
$\vv{BD} = -3 \vv{AB}  +  2 \vv{CA}$%
\end{center}%

\exo{Vecteurs et coordonnées.}%
Dans le plan muni d'un repère $\left( {{\mathrm{O}};\vec{\imath}, 
\vec{\jmath}} \right)$, on considère les points $A\left(-3; -2\right)$, $B\left(-1; 5
\right)$ et $C\left(4; 3\right)$.\\%
Calculez les coordonnées du point D tel que :%
\begin{center}%
$\vv{BD} = 2 \vv{AB}  -  4 \vv{CA}$%
\end{center}%

\exo{Vecteurs et coordonnées.}%
Dans le plan muni d'un repère $\left( {{\mathrm{O}};\vec{\imath}, 
\vec{\jmath}} \right)$, on considère les points $A\left(-1; -5\right)$, $B\left(-5; 2
\right)$ et $C\left(2; 1\right)$.\\%
Calculez les coordonnées du point D tel que :%
\begin{center}%
$\vv{BD} = -5 \vv{AB}  +  3 \vv{CA}$%
\end{center}%

\exo{Vecteurs et coordonnées.}%
Dans le plan muni d'un repère $\left( {{\mathrm{O}};\vec{\imath}, 
\vec{\jmath}} \right)$, on considère les points $A\left(-4; -5\right)$, $B\left(-1; 2
\right)$ et $C\left(2; 3\right)$.\\%
Calculez les coordonnées du point D tel que :%
\begin{center}%
$\vv{BD} = 5 \vv{AB}  +  4 \vv{CA}$%
\end{center}%

\exo{Vecteurs et coordonnées.}%
Dans le plan muni d'un repère $\left( {{\mathrm{O}};\vec{\imath}, 
\vec{\jmath}} \right)$, on considère les points $A\left(-2; -4\right)$, $B\left(-3; 4
\right)$ et $C\left(2; 2\right)$.\\%
Calculez les coordonnées du point D tel que :%
\begin{center}%
$\vv{BD} = -2 \vv{AB}  +  5 \vv{CA}$%
\end{center}%

\exo{Vecteurs et coordonnées.}%
Dans le plan muni d'un repère $\left( {{\mathrm{O}};\vec{\imath}, 
\vec{\jmath}} \right)$, on considère les points $A\left(-2; -1\right)$, $B\left(-4; 4
\right)$ et $C\left(3; 2\right)$.\\%
Calculez les coordonnées du point D tel que :%
\begin{center}%
$\vv{BD} = -2 \vv{AB}  -  4 \vv{CA}$%
\end{center}%

\exo{Vecteurs et coordonnées.}%
Dans le plan muni d'un repère $\left( {{\mathrm{O}};\vec{\imath}, 
\vec{\jmath}} \right)$, on considère les points $A\left(-3; -2\right)$, $B\left(-5; 1
\right)$ et $C\left(5; 1\right)$.\\%
Calculez les coordonnées du point D tel que :%
\begin{center}%
$\vv{BD} = 3 \vv{AB}  +  5 \vv{CA}$%
\end{center}%

\exo{Vecteurs et coordonnées.}%
Dans le plan muni d'un repère $\left( {{\mathrm{O}};\vec{\imath}, 
\vec{\jmath}} \right)$, on considère les points $A\left(-4; -5\right)$, $B\left(-2; 3
\right)$ et $C\left(1; 5\right)$.\\%
Calculez les coordonnées du point D tel que :%
\begin{center}%
$\vv{BD} = -2 \vv{AB}  +  3 \vv{CA}$%
\end{center}%

\exo{Vecteurs et coordonnées.}%
Dans le plan muni d'un repère $\left( {{\mathrm{O}};\vec{\imath}, 
\vec{\jmath}} \right)$, on considère les points $A\left(-3; -4\right)$, $B\left(-2; 4
\right)$ et $C\left(4; 2\right)$.\\%
Calculez les coordonnées du point D tel que :%
\begin{center}%
$\vv{BD} = -2 \vv{AB}  +  4 \vv{CA}$%
\end{center}%

\exo{Vecteurs et coordonnées.}%
Dans le plan muni d'un repère $\left( {{\mathrm{O}};\vec{\imath}, 
\vec{\jmath}} \right)$, on considère les points $A\left(-3; -4\right)$, $B\left(-4; 1
\right)$ et $C\left(2; 2\right)$.\\%
Calculez les coordonnées du point D tel que :%
\begin{center}%
$\vv{BD} = 2 \vv{AB}  -  4 \vv{CA}$%
\end{center}%

\exo{Vecteurs et coordonnées.}%
Dans le plan muni d'un repère $\left( {{\mathrm{O}};\vec{\imath}, 
\vec{\jmath}} \right)$, on considère les points $A\left(-3; -3\right)$, $B\left(-2; 2
\right)$ et $C\left(3; 1\right)$.\\%
Calculez les coordonnées du point D tel que :%
\begin{center}%
$\vv{BD} = -5 \vv{AB}  +  2 \vv{CA}$%
\end{center}%

\exo{Vecteurs et coordonnées.}%
Dans le plan muni d'un repère $\left( {{\mathrm{O}};\vec{\imath}, 
\vec{\jmath}} \right)$, on considère les points $A\left(-3; -1\right)$, $B\left(-2; 1
\right)$ et $C\left(3; 5\right)$.\\%
Calculez les coordonnées du point D tel que :%
\begin{center}%
$\vv{BD} = -5 \vv{AB}  +  4 \vv{CA}$%
\end{center}%

\exo{Vecteurs et coordonnées.}%
Dans le plan muni d'un repère $\left( {{\mathrm{O}};\vec{\imath}, 
\vec{\jmath}} \right)$, on considère les points $A\left(-5; -3\right)$, $B\left(-2; 5
\right)$ et $C\left(4; 2\right)$.\\%
Calculez les coordonnées du point D tel que :%
\begin{center}%
$\vv{BD} = -2 \vv{AB}  +  2 \vv{CA}$%
\end{center}%

\exo{Vecteurs et coordonnées.}%
Dans le plan muni d'un repère $\left( {{\mathrm{O}};\vec{\imath}, 
\vec{\jmath}} \right)$, on considère les points $A\left(-3; -2\right)$, $B\left(-2; 2
\right)$ et $C\left(2; 5\right)$.\\%
Calculez les coordonnées du point D tel que :%
\begin{center}%
$\vv{BD} = -5 \vv{AB}  +  3 \vv{CA}$%
\end{center}%

\exo{Vecteurs et coordonnées.}%
Dans le plan muni d'un repère $\left( {{\mathrm{O}};\vec{\imath}, 
\vec{\jmath}} \right)$, on considère les points $A\left(-3; -3\right)$, $B\left(-5; 3
\right)$ et $C\left(3; 4\right)$.\\%
Calculez les coordonnées du point D tel que :%
\begin{center}%
$\vv{BD} = 4 \vv{AB}  +  2 \vv{CA}$%
\end{center}%

\exo{Vecteurs et coordonnées.}%
Dans le plan muni d'un repère $\left( {{\mathrm{O}};\vec{\imath}, 
\vec{\jmath}} \right)$, on considère les points $A\left(-4; -2\right)$, $B\left(-4; 2
\right)$ et $C\left(5; 3\right)$.\\%
Calculez les coordonnées du point D tel que :%
\begin{center}%
$\vv{BD} = -4 \vv{AB}  +  3 \vv{CA}$%
\end{center}%

\exo{Vecteurs et coordonnées.}%
Dans le plan muni d'un repère $\left( {{\mathrm{O}};\vec{\imath}, 
\vec{\jmath}} \right)$, on considère les points $A\left(-1; -4\right)$, $B\left(-3; 1
\right)$ et $C\left(1; 5\right)$.\\%
Calculez les coordonnées du point D tel que :%
\begin{center}%
$\vv{BD} = 4 \vv{AB}  -  5 \vv{CA}$%
\end{center}%

\exo{Vecteurs et coordonnées.}%
Dans le plan muni d'un repère $\left( {{\mathrm{O}};\vec{\imath}, 
\vec{\jmath}} \right)$, on considère les points $A\left(-4; -4\right)$, $B\left(-3; 4
\right)$ et $C\left(4; 5\right)$.\\%
Calculez les coordonnées du point D tel que :%
\begin{center}%
$\vv{BD} = 5 \vv{AB}  +  2 \vv{CA}$%
\end{center}%

\exo{Vecteurs et coordonnées.}%
Dans le plan muni d'un repère $\left( {{\mathrm{O}};\vec{\imath}, 
\vec{\jmath}} \right)$, on considère les points $A\left(-5; -5\right)$, $B\left(-5; 4
\right)$ et $C\left(5; 3\right)$.\\%
Calculez les coordonnées du point D tel que :%
\begin{center}%
$\vv{BD} = 4 \vv{AB}  +  5 \vv{CA}$%
\end{center}%

\exo{Vecteurs et coordonnées.}%
Dans le plan muni d'un repère $\left( {{\mathrm{O}};\vec{\imath}, 
\vec{\jmath}} \right)$, on considère les points $A\left(-3; -2\right)$, $B\left(-5; 2
\right)$ et $C\left(5; 1\right)$.\\%
Calculez les coordonnées du point D tel que :%
\begin{center}%
$\vv{BD} = 4 \vv{AB}  +  3 \vv{CA}$%
\end{center}%

\exo{Vecteurs et coordonnées.}%
Dans le plan muni d'un repère $\left( {{\mathrm{O}};\vec{\imath}, 
\vec{\jmath}} \right)$, on considère les points $A\left(-4; -3\right)$, $B\left(-5; 5
\right)$ et $C\left(2; 3\right)$.\\%
Calculez les coordonnées du point D tel que :%
\begin{center}%
$\vv{BD} = 2 \vv{AB}  -  2 \vv{CA}$%
\end{center}%

\exo{Vecteurs et coordonnées.}%
Dans le plan muni d'un repère $\left( {{\mathrm{O}};\vec{\imath}, 
\vec{\jmath}} \right)$, on considère les points $A\left(-2; -5\right)$, $B\left(-4; 2
\right)$ et $C\left(3; 4\right)$.\\%
Calculez les coordonnées du point D tel que :%
\begin{center}%
$\vv{BD} = -5 \vv{AB}  -  5 \vv{CA}$%
\end{center}%

\exo{Vecteurs et coordonnées.}%
Dans le plan muni d'un repère $\left( {{\mathrm{O}};\vec{\imath}, 
\vec{\jmath}} \right)$, on considère les points $A\left(-4; -4\right)$, $B\left(-3; 5
\right)$ et $C\left(5; 1\right)$.\\%
Calculez les coordonnées du point D tel que :%
\begin{center}%
$\vv{BD} = 3 \vv{AB}  +  4 \vv{CA}$%
\end{center}%

\exo{Vecteurs et coordonnées.}%
Dans le plan muni d'un repère $\left( {{\mathrm{O}};\vec{\imath}, 
\vec{\jmath}} \right)$, on considère les points $A\left(-2; -2\right)$, $B\left(-4; 3
\right)$ et $C\left(4; 5\right)$.\\%
Calculez les coordonnées du point D tel que :%
\begin{center}%
$\vv{BD} = -4 \vv{AB}  +  4 \vv{CA}$%
\end{center}%

\exo{Vecteurs et coordonnées.}%
Dans le plan muni d'un repère $\left( {{\mathrm{O}};\vec{\imath}, 
\vec{\jmath}} \right)$, on considère les points $A\left(-1; -3\right)$, $B\left(-1; 2
\right)$ et $C\left(5; 5\right)$.\\%
Calculez les coordonnées du point D tel que :%
\begin{center}%
$\vv{BD} = -3 \vv{AB}  -  5 \vv{CA}$%
\end{center}%

\exo{Vecteurs et coordonnées.}%
Dans le plan muni d'un repère $\left( {{\mathrm{O}};\vec{\imath}, 
\vec{\jmath}} \right)$, on considère les points $A\left(-5; -2\right)$, $B\left(-5; 5
\right)$ et $C\left(3; 1\right)$.\\%
Calculez les coordonnées du point D tel que :%
\begin{center}%
$\vv{BD} = 5 \vv{AB}  +  4 \vv{CA}$%
\end{center}%

\exo{Vecteurs et coordonnées.}%
Dans le plan muni d'un repère $\left( {{\mathrm{O}};\vec{\imath}, 
\vec{\jmath}} \right)$, on considère les points $A\left(-4; -3\right)$, $B\left(-2; 1
\right)$ et $C\left(4; 3\right)$.\\%
Calculez les coordonnées du point D tel que :%
\begin{center}%
$\vv{BD} = -3 \vv{AB}  -  4 \vv{CA}$%
\end{center}%

\end{document}