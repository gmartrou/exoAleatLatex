\documentclass[10pt, a4paper]{article}
\usepackage[utf8x]{inputenc}
\usepackage[french]{babel}
\usepackage[T1]{fontenc}
\usepackage{adjustbox}
\newcounter{nexo}           % déclaration du numéro d'exo
\setcounter{nexo}{0}        % initialisation du numero
\newcommand{\exo}[1]{
  \refstepcounter{nexo} 
  \par{{\section*{Exercice \arabic{nexo} : #1}}}\noindent}
\newcommand{\cor}[1]{
  \refstepcounter{nexo}
  \par{{\section*{Correction \arabic{nexo} : #1}}}\noindent}
\newcommand{\vv}[1]{\overrightarrow{#1}}
\newcommand{\coord}[3]{#1\begin{pmatrix}
  #2 \\
  #3
  \end{pmatrix}}
\newcommand{\coordv}[3]{\vv{#1}\begin{pmatrix}
  #2 \\
  #3
  \end{pmatrix}}
\usepackage{graphicx}
\usepackage{eurosym}
\usepackage{textcomp}
\usepackage{amsmath}
\usepackage{geometry}
\geometry{tmargin=0.6cm,lmargin=1cm,rmargin=1cm,bmargin=1.5cm}
\usepackage{xcolor}
\usepackage{dsfont}
\newcommand{\R}{\mathds {R}}
\usepackage{tikz,tkz-tab}
\usetikzlibrary{shapes.misc}
\tikzset{cross/.style={cross out, draw=red, minimum size=2*(#1-\pgflinewidth), inner sep=0pt, outer sep=0pt},
%default radius will be 1pt. 
cross/.default={1pt}}
\usepackage{pgfplots}
\pgfplotsset{compat=1.16}
\usepackage{interval}
\setlength\parindent{0pt}
\usepackage{enumitem}
%---- Style de l'entête pour le grade 10 LLG Paris-Abu Dhabi-----
%\setlength{\parindent}{0.5cm}
\newcommand{\enteteLSMI}[2]
{
  % Lieu - Année
  \noindent{\underline{\LARGE \textbf{Lycée Stendhal}} \hfill \large \textbf{Mathématiques, Classe de 4$^{\text{ème}}$}


  %  Module - Type de feuille
  \noindent{\ \ \LARGE AEFE - Milan  \large \hfill {#1}}\\
  \noindent{\phantom{Name : \dotfill}}\hfill }  
 \vspace*{-0.3cm}
  \hrule
  \begin{center} \textbf{\textsf{\Large #2 }} \end{center}\vspace*{-0.12cm}
  \hrule
  \vspace*{0.25cm}
}%
\begin{document}%

\enteteLSMI{\today}{Correction de l'interrogation n°10 : les tableaux de signes : sujet A}%

\cor{Vecteurs et coordonnées.}%
\begin{enumerate}%
\item On calcule les coordonnées des vecteurs $\vv{AB}$ et $\vv{CA}$\\%
$\coordv{AB}{x_B - x_A}{y_B - y_A}$. Donc, $\coordv{AB}{-2 - (-3)}{5 - -5}$. 
Ainsi, $\coordv{AB}{1}{10}$\\%

$\coordv{CA}{x_A - x_C}{y_A - y_C}$. Donc, $\coordv{CA}{-3 - 2}{-5 - 2}$. 
Ainsi, $\coordv{CA}{-5}{-7}$\\%

\item On calcule maintenant, grâce à ces vecteurs, les coordonnées du vecteurs $\vv{BD}$\\%
$\vv{BD}=-3 \vv{AB}  +  5 \vv{CA}$.
Donc, $\coordv{BD}{-3 \times 1  +  5 \times (-5)}{
-3 \times 10  +  5 \times(-7)}$.
Soit, $\coordv{BD}{-28}{-65}$.\\%
\item À partir des coordonnées de B et $\vv{BD}$, on calcule ceux du point D.\\%
On a $\coord{B}{-2}{5}$ et $\coordv{BD}{-28}{-65}$. 
Donc $\coord{D}{-2  -  28}{5  -  65}$\\%

Et, finalement, $\coord{D}{-30}{-60}$%
\end{enumerate}%

\cor{Vecteurs et coordonnées.}%
\begin{enumerate}%
\item On calcule les coordonnées des vecteurs $\vv{AB}$ et $\vv{CA}$\\%
$\coordv{AB}{x_B - x_A}{y_B - y_A}$. Donc, $\coordv{AB}{-3 - (-2)}{4 - -5}$. 
Ainsi, $\coordv{AB}{-1}{9}$\\%

$\coordv{CA}{x_A - x_C}{y_A - y_C}$. Donc, $\coordv{CA}{-2 - 2}{-5 - 4}$. 
Ainsi, $\coordv{CA}{-4}{-9}$\\%

\item On calcule maintenant, grâce à ces vecteurs, les coordonnées du vecteurs $\vv{BD}$\\%
$\vv{BD}=-5 \vv{AB}  -  4 \vv{CA}$.
Donc, $\coordv{BD}{-5 \times -1  -  4 \times-4}{
-5 \times 9  -  4 \times (-9)}$. 
Soit, $\coordv{BD}{21}{-9}$.\\%
\item À partir des coordonnées de B et $\vv{BD}$, on calcule ceux du point D.\\%
On a $\coord{B}{-3}{4}$ et $\coordv{BD}{21}{-9}$. 
Donc $\coord{D}{-3  +  21}{4  -  9}$\\%

Et, finalement, $\coord{D}{18}{-5}$%
\end{enumerate}%

\cor{Vecteurs et coordonnées.}%
\begin{enumerate}%
\item On calcule les coordonnées des vecteurs $\vv{AB}$ et $\vv{CA}$\\%
$\coordv{AB}{x_B - x_A}{y_B - y_A}$. Donc, $\coordv{AB}{-4 - (-1)}{1 - -2}$. 
Ainsi, $\coordv{AB}{-3}{3}$\\%

$\coordv{CA}{x_A - x_C}{y_A - y_C}$. Donc, $\coordv{CA}{-1 - 1}{-2 - 2}$. 
Ainsi, $\coordv{CA}{-2}{-4}$\\%

\item On calcule maintenant, grâce à ces vecteurs, les coordonnées du vecteurs $\vv{BD}$\\%
$\vv{BD}=4 \vv{AB}  -  2 \vv{CA}$.
Donc, $\coordv{BD}{4 \times -3  -  2 \times-2}{
4 \times 3  -  2 \times (-4)}$. 
Soit, $\coordv{BD}{-8}{20}$.\\%
\item À partir des coordonnées de B et $\vv{BD}$, on calcule ceux du point D.\\%
On a $\coord{B}{-4}{1}$ et $\coordv{BD}{-8}{20}$. 
Donc $\coord{D}{-4  -  8}{1  +  20}$\\%

Et, finalement, $\coord{D}{-12}{21}$%
\end{enumerate}%

\cor{Vecteurs et coordonnées.}%
\begin{enumerate}%
\item On calcule les coordonnées des vecteurs $\vv{AB}$ et $\vv{CA}$\\%
$\coordv{AB}{x_B - x_A}{y_B - y_A}$. Donc, $\coordv{AB}{-3 - (-4)}{5 - -3}$. 
Ainsi, $\coordv{AB}{1}{8}$\\%

$\coordv{CA}{x_A - x_C}{y_A - y_C}$. Donc, $\coordv{CA}{-4 - 3}{-3 - 2}$. 
Ainsi, $\coordv{CA}{-7}{-5}$\\%

\item On calcule maintenant, grâce à ces vecteurs, les coordonnées du vecteurs $\vv{BD}$\\%
$\vv{BD}=-3 \vv{AB}  +  2 \vv{CA}$.
Donc, $\coordv{BD}{-3 \times 1  +  2 \times (-7)}{
-3 \times 8  +  2 \times(-5)}$.
Soit, $\coordv{BD}{11}{-14}$.\\%
\item À partir des coordonnées de B et $\vv{BD}$, on calcule ceux du point D.\\%
On a $\coord{B}{-3}{5}$ et $\coordv{BD}{11}{-14}$. 
Donc $\coord{D}{-3  +  11}{5  -  14}$\\%

Et, finalement, $\coord{D}{8}{-9}$%
\end{enumerate}%

\cor{Vecteurs et coordonnées.}%
\begin{enumerate}%
\item On calcule les coordonnées des vecteurs $\vv{AB}$ et $\vv{CA}$\\%
$\coordv{AB}{x_B - x_A}{y_B - y_A}$. Donc, $\coordv{AB}{-3 - (-5)}{5 - -3}$. 
Ainsi, $\coordv{AB}{2}{8}$\\%

$\coordv{CA}{x_A - x_C}{y_A - y_C}$. Donc, $\coordv{CA}{-5 - 2}{-3 - 2}$. 
Ainsi, $\coordv{CA}{-7}{-5}$\\%

\item On calcule maintenant, grâce à ces vecteurs, les coordonnées du vecteurs $\vv{BD}$\\%
$\vv{BD}=-4 \vv{AB}  -  2 \vv{CA}$.
Donc, $\coordv{BD}{-4 \times 2  -  2 \times (-7)}{
-4 \times 8  -  2 \times(-5)}$.
Soit, $\coordv{BD}{6}{-22}$.\\%
\item À partir des coordonnées de B et $\vv{BD}$, on calcule ceux du point D.\\%
On a $\coord{B}{-3}{5}$ et $\coordv{BD}{6}{-22}$. 
Donc $\coord{D}{-3  +  6}{5  -  22}$\\%

Et, finalement, $\coord{D}{3}{-17}$%
\end{enumerate}%

\cor{Vecteurs et coordonnées.}%
\begin{enumerate}%
\item On calcule les coordonnées des vecteurs $\vv{AB}$ et $\vv{CA}$\\%
$\coordv{AB}{x_B - x_A}{y_B - y_A}$. Donc, $\coordv{AB}{-2 - (-1)}{2 - -5}$. 
Ainsi, $\coordv{AB}{-1}{7}$\\%

$\coordv{CA}{x_A - x_C}{y_A - y_C}$. Donc, $\coordv{CA}{-1 - 4}{-5 - 4}$. 
Ainsi, $\coordv{CA}{-5}{-9}$\\%

\item On calcule maintenant, grâce à ces vecteurs, les coordonnées du vecteurs $\vv{BD}$\\%
$\vv{BD}=4 \vv{AB}  +  5 \vv{CA}$.
Donc, $\coordv{BD}{4 \times -1  +  5 \times-5}{
4 \times 7  +  5 \times (-9)}$. 
Soit, $\coordv{BD}{21}{73}$.\\%
\item À partir des coordonnées de B et $\vv{BD}$, on calcule ceux du point D.\\%
On a $\coord{B}{-2}{2}$ et $\coordv{BD}{21}{73}$. 
Donc $\coord{D}{-2  +  21}{2  +  73}$\\%

Et, finalement, $\coord{D}{19}{75}$%
\end{enumerate}%

\cor{Vecteurs et coordonnées.}%
\begin{enumerate}%
\item On calcule les coordonnées des vecteurs $\vv{AB}$ et $\vv{CA}$\\%
$\coordv{AB}{x_B - x_A}{y_B - y_A}$. Donc, $\coordv{AB}{-3 - (-2)}{4 - -5}$. 
Ainsi, $\coordv{AB}{-1}{9}$\\%

$\coordv{CA}{x_A - x_C}{y_A - y_C}$. Donc, $\coordv{CA}{-2 - 1}{-5 - 3}$. 
Ainsi, $\coordv{CA}{-3}{-8}$\\%

\item On calcule maintenant, grâce à ces vecteurs, les coordonnées du vecteurs $\vv{BD}$\\%
$\vv{BD}=5 \vv{AB}  +  4 \vv{CA}$.
Donc, $\coordv{BD}{5 \times -1  +  4 \times-3}{
5 \times 9  +  4 \times (-8)}$. 
Soit, $\coordv{BD}{-17}{13}$.\\%
\item À partir des coordonnées de B et $\vv{BD}$, on calcule ceux du point D.\\%
On a $\coord{B}{-3}{4}$ et $\coordv{BD}{-17}{13}$. 
Donc $\coord{D}{-3  -  17}{4  +  13}$\\%

Et, finalement, $\coord{D}{-20}{17}$%
\end{enumerate}%

\cor{Vecteurs et coordonnées.}%
\begin{enumerate}%
\item On calcule les coordonnées des vecteurs $\vv{AB}$ et $\vv{CA}$\\%
$\coordv{AB}{x_B - x_A}{y_B - y_A}$. Donc, $\coordv{AB}{-4 - (-1)}{3 - -5}$. 
Ainsi, $\coordv{AB}{-3}{8}$\\%

$\coordv{CA}{x_A - x_C}{y_A - y_C}$. Donc, $\coordv{CA}{-1 - 5}{-5 - 3}$. 
Ainsi, $\coordv{CA}{-6}{-8}$\\%

\item On calcule maintenant, grâce à ces vecteurs, les coordonnées du vecteurs $\vv{BD}$\\%
$\vv{BD}=2 \vv{AB}  -  3 \vv{CA}$.
Donc, $\coordv{BD}{2 \times -3  -  3 \times-6}{
2 \times 8  -  3 \times (-8)}$. 
Soit, $\coordv{BD}{12}{40}$.\\%
\item À partir des coordonnées de B et $\vv{BD}$, on calcule ceux du point D.\\%
On a $\coord{B}{-4}{3}$ et $\coordv{BD}{12}{40}$. 
Donc $\coord{D}{-4  +  12}{3  +  40}$\\%

Et, finalement, $\coord{D}{8}{43}$%
\end{enumerate}%

\cor{Vecteurs et coordonnées.}%
\begin{enumerate}%
\item On calcule les coordonnées des vecteurs $\vv{AB}$ et $\vv{CA}$\\%
$\coordv{AB}{x_B - x_A}{y_B - y_A}$. Donc, $\coordv{AB}{-1 - (-1)}{3 - -4}$. 
Ainsi, $\coordv{AB}{0}{7}$\\%

$\coordv{CA}{x_A - x_C}{y_A - y_C}$. Donc, $\coordv{CA}{-1 - 2}{-4 - 5}$. 
Ainsi, $\coordv{CA}{-3}{-9}$\\%

\item On calcule maintenant, grâce à ces vecteurs, les coordonnées du vecteurs $\vv{BD}$\\%
$\vv{BD}=-3 \vv{AB}  -  5 \vv{CA}$.
Donc, $\coordv{BD}{-3 \times 0  -  5 \times-3}{
-3 \times 7  -  5 \times (-9)}$. 
Soit, $\coordv{BD}{15}{24}$.\\%
\item À partir des coordonnées de B et $\vv{BD}$, on calcule ceux du point D.\\%
On a $\coord{B}{-1}{3}$ et $\coordv{BD}{15}{24}$. 
Donc $\coord{D}{-1  +  15}{3  +  24}$\\%

Et, finalement, $\coord{D}{14}{27}$%
\end{enumerate}%

\cor{Vecteurs et coordonnées.}%
\begin{enumerate}%
\item On calcule les coordonnées des vecteurs $\vv{AB}$ et $\vv{CA}$\\%
$\coordv{AB}{x_B - x_A}{y_B - y_A}$. Donc, $\coordv{AB}{-2 - (-5)}{4 - -5}$. 
Ainsi, $\coordv{AB}{3}{9}$\\%

$\coordv{CA}{x_A - x_C}{y_A - y_C}$. Donc, $\coordv{CA}{-5 - 1}{-5 - 1}$. 
Ainsi, $\coordv{CA}{-6}{-6}$\\%

\item On calcule maintenant, grâce à ces vecteurs, les coordonnées du vecteurs $\vv{BD}$\\%
$\vv{BD}=-4 \vv{AB}  +  5 \vv{CA}$.
Donc, $\coordv{BD}{-4 \times 3  +  5 \times (-6)}{
-4 \times 9  +  5 \times(-6)}$.
Soit, $\coordv{BD}{-42}{-66}$.\\%
\item À partir des coordonnées de B et $\vv{BD}$, on calcule ceux du point D.\\%
On a $\coord{B}{-2}{4}$ et $\coordv{BD}{-42}{-66}$. 
Donc $\coord{D}{-2  -  42}{4  -  66}$\\%

Et, finalement, $\coord{D}{-44}{-62}$%
\end{enumerate}%

\cor{Vecteurs et coordonnées.}%
\begin{enumerate}%
\item On calcule les coordonnées des vecteurs $\vv{AB}$ et $\vv{CA}$\\%
$\coordv{AB}{x_B - x_A}{y_B - y_A}$. Donc, $\coordv{AB}{-5 - (-1)}{4 - -5}$. 
Ainsi, $\coordv{AB}{-4}{9}$\\%

$\coordv{CA}{x_A - x_C}{y_A - y_C}$. Donc, $\coordv{CA}{-1 - 4}{-5 - 4}$. 
Ainsi, $\coordv{CA}{-5}{-9}$\\%

\item On calcule maintenant, grâce à ces vecteurs, les coordonnées du vecteurs $\vv{BD}$\\%
$\vv{BD}=3 \vv{AB}  +  2 \vv{CA}$.
Donc, $\coordv{BD}{3 \times -4  +  2 \times-5}{
3 \times 9  +  2 \times (-9)}$. 
Soit, $\coordv{BD}{-22}{9}$.\\%
\item À partir des coordonnées de B et $\vv{BD}$, on calcule ceux du point D.\\%
On a $\coord{B}{-5}{4}$ et $\coordv{BD}{-22}{9}$. 
Donc $\coord{D}{-5  -  22}{4  +  9}$\\%

Et, finalement, $\coord{D}{-27}{13}$%
\end{enumerate}%

\cor{Vecteurs et coordonnées.}%
\begin{enumerate}%
\item On calcule les coordonnées des vecteurs $\vv{AB}$ et $\vv{CA}$\\%
$\coordv{AB}{x_B - x_A}{y_B - y_A}$. Donc, $\coordv{AB}{-5 - (-1)}{2 - -2}$. 
Ainsi, $\coordv{AB}{-4}{4}$\\%

$\coordv{CA}{x_A - x_C}{y_A - y_C}$. Donc, $\coordv{CA}{-1 - 5}{-2 - 1}$. 
Ainsi, $\coordv{CA}{-6}{-3}$\\%

\item On calcule maintenant, grâce à ces vecteurs, les coordonnées du vecteurs $\vv{BD}$\\%
$\vv{BD}=-3 \vv{AB}  +  5 \vv{CA}$.
Donc, $\coordv{BD}{-3 \times -4  +  5 \times-6}{
-3 \times 4  +  5 \times (-3)}$. 
Soit, $\coordv{BD}{-18}{-27}$.\\%
\item À partir des coordonnées de B et $\vv{BD}$, on calcule ceux du point D.\\%
On a $\coord{B}{-5}{2}$ et $\coordv{BD}{-18}{-27}$. 
Donc $\coord{D}{-5  -  18}{2  -  27}$\\%

Et, finalement, $\coord{D}{-23}{-25}$%
\end{enumerate}%

\cor{Vecteurs et coordonnées.}%
\begin{enumerate}%
\item On calcule les coordonnées des vecteurs $\vv{AB}$ et $\vv{CA}$\\%
$\coordv{AB}{x_B - x_A}{y_B - y_A}$. Donc, $\coordv{AB}{-2 - (-4)}{3 - -1}$. 
Ainsi, $\coordv{AB}{2}{4}$\\%

$\coordv{CA}{x_A - x_C}{y_A - y_C}$. Donc, $\coordv{CA}{-4 - 2}{-1 - 1}$. 
Ainsi, $\coordv{CA}{-6}{-2}$\\%

\item On calcule maintenant, grâce à ces vecteurs, les coordonnées du vecteurs $\vv{BD}$\\%
$\vv{BD}=2 \vv{AB}  +  3 \vv{CA}$.
Donc, $\coordv{BD}{2 \times 2  +  3 \times (-6)}{
2 \times 4  +  3 \times(-2)}$.
Soit, $\coordv{BD}{22}{14}$.\\%
\item À partir des coordonnées de B et $\vv{BD}$, on calcule ceux du point D.\\%
On a $\coord{B}{-2}{3}$ et $\coordv{BD}{22}{14}$. 
Donc $\coord{D}{-2  +  22}{3  +  14}$\\%

Et, finalement, $\coord{D}{20}{17}$%
\end{enumerate}%

\cor{Vecteurs et coordonnées.}%
\begin{enumerate}%
\item On calcule les coordonnées des vecteurs $\vv{AB}$ et $\vv{CA}$\\%
$\coordv{AB}{x_B - x_A}{y_B - y_A}$. Donc, $\coordv{AB}{-2 - (-4)}{2 - -5}$. 
Ainsi, $\coordv{AB}{2}{7}$\\%

$\coordv{CA}{x_A - x_C}{y_A - y_C}$. Donc, $\coordv{CA}{-4 - 1}{-5 - 1}$. 
Ainsi, $\coordv{CA}{-5}{-6}$\\%

\item On calcule maintenant, grâce à ces vecteurs, les coordonnées du vecteurs $\vv{BD}$\\%
$\vv{BD}=-3 \vv{AB}  +  2 \vv{CA}$.
Donc, $\coordv{BD}{-3 \times 2  +  2 \times (-5)}{
-3 \times 7  +  2 \times(-6)}$.
Soit, $\coordv{BD}{-16}{-33}$.\\%
\item À partir des coordonnées de B et $\vv{BD}$, on calcule ceux du point D.\\%
On a $\coord{B}{-2}{2}$ et $\coordv{BD}{-16}{-33}$. 
Donc $\coord{D}{-2  -  16}{2  -  33}$\\%

Et, finalement, $\coord{D}{-18}{-31}$%
\end{enumerate}%

\cor{Vecteurs et coordonnées.}%
\begin{enumerate}%
\item On calcule les coordonnées des vecteurs $\vv{AB}$ et $\vv{CA}$\\%
$\coordv{AB}{x_B - x_A}{y_B - y_A}$. Donc, $\coordv{AB}{-1 - (-3)}{5 - -2}$. 
Ainsi, $\coordv{AB}{2}{7}$\\%

$\coordv{CA}{x_A - x_C}{y_A - y_C}$. Donc, $\coordv{CA}{-3 - 4}{-2 - 3}$. 
Ainsi, $\coordv{CA}{-7}{-5}$\\%

\item On calcule maintenant, grâce à ces vecteurs, les coordonnées du vecteurs $\vv{BD}$\\%
$\vv{BD}=2 \vv{AB}  -  4 \vv{CA}$.
Donc, $\coordv{BD}{2 \times 2  -  4 \times (-7)}{
2 \times 7  -  4 \times(-5)}$.
Soit, $\coordv{BD}{32}{34}$.\\%
\item À partir des coordonnées de B et $\vv{BD}$, on calcule ceux du point D.\\%
On a $\coord{B}{-1}{5}$ et $\coordv{BD}{32}{34}$. 
Donc $\coord{D}{-1  +  32}{5  +  34}$\\%

Et, finalement, $\coord{D}{31}{39}$%
\end{enumerate}%

\cor{Vecteurs et coordonnées.}%
\begin{enumerate}%
\item On calcule les coordonnées des vecteurs $\vv{AB}$ et $\vv{CA}$\\%
$\coordv{AB}{x_B - x_A}{y_B - y_A}$. Donc, $\coordv{AB}{-5 - (-1)}{2 - -5}$. 
Ainsi, $\coordv{AB}{-4}{7}$\\%

$\coordv{CA}{x_A - x_C}{y_A - y_C}$. Donc, $\coordv{CA}{-1 - 2}{-5 - 1}$. 
Ainsi, $\coordv{CA}{-3}{-6}$\\%

\item On calcule maintenant, grâce à ces vecteurs, les coordonnées du vecteurs $\vv{BD}$\\%
$\vv{BD}=-5 \vv{AB}  +  3 \vv{CA}$.
Donc, $\coordv{BD}{-5 \times -4  +  3 \times-3}{
-5 \times 7  +  3 \times (-6)}$. 
Soit, $\coordv{BD}{11}{-53}$.\\%
\item À partir des coordonnées de B et $\vv{BD}$, on calcule ceux du point D.\\%
On a $\coord{B}{-5}{2}$ et $\coordv{BD}{11}{-53}$. 
Donc $\coord{D}{-5  +  11}{2  -  53}$\\%

Et, finalement, $\coord{D}{6}{-51}$%
\end{enumerate}%

\cor{Vecteurs et coordonnées.}%
\begin{enumerate}%
\item On calcule les coordonnées des vecteurs $\vv{AB}$ et $\vv{CA}$\\%
$\coordv{AB}{x_B - x_A}{y_B - y_A}$. Donc, $\coordv{AB}{-1 - (-4)}{2 - -5}$. 
Ainsi, $\coordv{AB}{3}{7}$\\%

$\coordv{CA}{x_A - x_C}{y_A - y_C}$. Donc, $\coordv{CA}{-4 - 2}{-5 - 3}$. 
Ainsi, $\coordv{CA}{-6}{-8}$\\%

\item On calcule maintenant, grâce à ces vecteurs, les coordonnées du vecteurs $\vv{BD}$\\%
$\vv{BD}=5 \vv{AB}  +  4 \vv{CA}$.
Donc, $\coordv{BD}{5 \times 3  +  4 \times (-6)}{
5 \times 7  +  4 \times(-8)}$.
Soit, $\coordv{BD}{-9}{3}$.\\%
\item À partir des coordonnées de B et $\vv{BD}$, on calcule ceux du point D.\\%
On a $\coord{B}{-1}{2}$ et $\coordv{BD}{-9}{3}$. 
Donc $\coord{D}{-1  -  9}{2  +  3}$\\%

Et, finalement, $\coord{D}{-10}{5}$%
\end{enumerate}%

\cor{Vecteurs et coordonnées.}%
\begin{enumerate}%
\item On calcule les coordonnées des vecteurs $\vv{AB}$ et $\vv{CA}$\\%
$\coordv{AB}{x_B - x_A}{y_B - y_A}$. Donc, $\coordv{AB}{-3 - (-2)}{4 - -4}$. 
Ainsi, $\coordv{AB}{-1}{8}$\\%

$\coordv{CA}{x_A - x_C}{y_A - y_C}$. Donc, $\coordv{CA}{-2 - 2}{-4 - 2}$. 
Ainsi, $\coordv{CA}{-4}{-6}$\\%

\item On calcule maintenant, grâce à ces vecteurs, les coordonnées du vecteurs $\vv{BD}$\\%
$\vv{BD}=-2 \vv{AB}  +  5 \vv{CA}$.
Donc, $\coordv{BD}{-2 \times -1  +  5 \times-4}{
-2 \times 8  +  5 \times (-6)}$. 
Soit, $\coordv{BD}{-18}{-46}$.\\%
\item À partir des coordonnées de B et $\vv{BD}$, on calcule ceux du point D.\\%
On a $\coord{B}{-3}{4}$ et $\coordv{BD}{-18}{-46}$. 
Donc $\coord{D}{-3  -  18}{4  -  46}$\\%

Et, finalement, $\coord{D}{-21}{-42}$%
\end{enumerate}%

\cor{Vecteurs et coordonnées.}%
\begin{enumerate}%
\item On calcule les coordonnées des vecteurs $\vv{AB}$ et $\vv{CA}$\\%
$\coordv{AB}{x_B - x_A}{y_B - y_A}$. Donc, $\coordv{AB}{-4 - (-2)}{4 - -1}$. 
Ainsi, $\coordv{AB}{-2}{5}$\\%

$\coordv{CA}{x_A - x_C}{y_A - y_C}$. Donc, $\coordv{CA}{-2 - 3}{-1 - 2}$. 
Ainsi, $\coordv{CA}{-5}{-3}$\\%

\item On calcule maintenant, grâce à ces vecteurs, les coordonnées du vecteurs $\vv{BD}$\\%
$\vv{BD}=-2 \vv{AB}  -  4 \vv{CA}$.
Donc, $\coordv{BD}{-2 \times -2  -  4 \times-5}{
-2 \times 5  -  4 \times (-3)}$. 
Soit, $\coordv{BD}{24}{2}$.\\%
\item À partir des coordonnées de B et $\vv{BD}$, on calcule ceux du point D.\\%
On a $\coord{B}{-4}{4}$ et $\coordv{BD}{24}{2}$. 
Donc $\coord{D}{-4  +  24}{4  +  2}$\\%

Et, finalement, $\coord{D}{20}{6}$%
\end{enumerate}%

\cor{Vecteurs et coordonnées.}%
\begin{enumerate}%
\item On calcule les coordonnées des vecteurs $\vv{AB}$ et $\vv{CA}$\\%
$\coordv{AB}{x_B - x_A}{y_B - y_A}$. Donc, $\coordv{AB}{-5 - (-3)}{1 - -2}$. 
Ainsi, $\coordv{AB}{-2}{3}$\\%

$\coordv{CA}{x_A - x_C}{y_A - y_C}$. Donc, $\coordv{CA}{-3 - 5}{-2 - 1}$. 
Ainsi, $\coordv{CA}{-8}{-3}$\\%

\item On calcule maintenant, grâce à ces vecteurs, les coordonnées du vecteurs $\vv{BD}$\\%
$\vv{BD}=3 \vv{AB}  +  5 \vv{CA}$.
Donc, $\coordv{BD}{3 \times -2  +  5 \times-8}{
3 \times 3  +  5 \times (-3)}$. 
Soit, $\coordv{BD}{-46}{-6}$.\\%
\item À partir des coordonnées de B et $\vv{BD}$, on calcule ceux du point D.\\%
On a $\coord{B}{-5}{1}$ et $\coordv{BD}{-46}{-6}$. 
Donc $\coord{D}{-5  -  46}{1  -  6}$\\%

Et, finalement, $\coord{D}{-51}{-5}$%
\end{enumerate}%

\cor{Vecteurs et coordonnées.}%
\begin{enumerate}%
\item On calcule les coordonnées des vecteurs $\vv{AB}$ et $\vv{CA}$\\%
$\coordv{AB}{x_B - x_A}{y_B - y_A}$. Donc, $\coordv{AB}{-2 - (-4)}{3 - -5}$. 
Ainsi, $\coordv{AB}{2}{8}$\\%

$\coordv{CA}{x_A - x_C}{y_A - y_C}$. Donc, $\coordv{CA}{-4 - 1}{-5 - 5}$. 
Ainsi, $\coordv{CA}{-5}{-10}$\\%

\item On calcule maintenant, grâce à ces vecteurs, les coordonnées du vecteurs $\vv{BD}$\\%
$\vv{BD}=-2 \vv{AB}  +  3 \vv{CA}$.
Donc, $\coordv{BD}{-2 \times 2  +  3 \times (-5)}{
-2 \times 8  +  3 \times(-10)}$.
Soit, $\coordv{BD}{-19}{-46}$.\\%
\item À partir des coordonnées de B et $\vv{BD}$, on calcule ceux du point D.\\%
On a $\coord{B}{-2}{3}$ et $\coordv{BD}{-19}{-46}$. 
Donc $\coord{D}{-2  -  19}{3  -  46}$\\%

Et, finalement, $\coord{D}{-21}{-43}$%
\end{enumerate}%

\cor{Vecteurs et coordonnées.}%
\begin{enumerate}%
\item On calcule les coordonnées des vecteurs $\vv{AB}$ et $\vv{CA}$\\%
$\coordv{AB}{x_B - x_A}{y_B - y_A}$. Donc, $\coordv{AB}{-2 - (-3)}{4 - -4}$. 
Ainsi, $\coordv{AB}{1}{8}$\\%

$\coordv{CA}{x_A - x_C}{y_A - y_C}$. Donc, $\coordv{CA}{-3 - 4}{-4 - 2}$. 
Ainsi, $\coordv{CA}{-7}{-6}$\\%

\item On calcule maintenant, grâce à ces vecteurs, les coordonnées du vecteurs $\vv{BD}$\\%
$\vv{BD}=-2 \vv{AB}  +  4 \vv{CA}$.
Donc, $\coordv{BD}{-2 \times 1  +  4 \times (-7)}{
-2 \times 8  +  4 \times(-6)}$.
Soit, $\coordv{BD}{26}{8}$.\\%
\item À partir des coordonnées de B et $\vv{BD}$, on calcule ceux du point D.\\%
On a $\coord{B}{-2}{4}$ et $\coordv{BD}{26}{8}$. 
Donc $\coord{D}{-2  +  26}{4  +  8}$\\%

Et, finalement, $\coord{D}{24}{12}$%
\end{enumerate}%

\cor{Vecteurs et coordonnées.}%
\begin{enumerate}%
\item On calcule les coordonnées des vecteurs $\vv{AB}$ et $\vv{CA}$\\%
$\coordv{AB}{x_B - x_A}{y_B - y_A}$. Donc, $\coordv{AB}{-4 - (-3)}{1 - -4}$. 
Ainsi, $\coordv{AB}{-1}{5}$\\%

$\coordv{CA}{x_A - x_C}{y_A - y_C}$. Donc, $\coordv{CA}{-3 - 2}{-4 - 2}$. 
Ainsi, $\coordv{CA}{-5}{-6}$\\%

\item On calcule maintenant, grâce à ces vecteurs, les coordonnées du vecteurs $\vv{BD}$\\%
$\vv{BD}=2 \vv{AB}  -  4 \vv{CA}$.
Donc, $\coordv{BD}{2 \times -1  -  4 \times-5}{
2 \times 5  -  4 \times (-6)}$. 
Soit, $\coordv{BD}{18}{34}$.\\%
\item À partir des coordonnées de B et $\vv{BD}$, on calcule ceux du point D.\\%
On a $\coord{B}{-4}{1}$ et $\coordv{BD}{18}{34}$. 
Donc $\coord{D}{-4  +  18}{1  +  34}$\\%

Et, finalement, $\coord{D}{14}{35}$%
\end{enumerate}%

\cor{Vecteurs et coordonnées.}%
\begin{enumerate}%
\item On calcule les coordonnées des vecteurs $\vv{AB}$ et $\vv{CA}$\\%
$\coordv{AB}{x_B - x_A}{y_B - y_A}$. Donc, $\coordv{AB}{-2 - (-3)}{2 - -3}$. 
Ainsi, $\coordv{AB}{1}{5}$\\%

$\coordv{CA}{x_A - x_C}{y_A - y_C}$. Donc, $\coordv{CA}{-3 - 3}{-3 - 1}$. 
Ainsi, $\coordv{CA}{-6}{-4}$\\%

\item On calcule maintenant, grâce à ces vecteurs, les coordonnées du vecteurs $\vv{BD}$\\%
$\vv{BD}=-5 \vv{AB}  +  2 \vv{CA}$.
Donc, $\coordv{BD}{-5 \times 1  +  2 \times (-6)}{
-5 \times 5  +  2 \times(-4)}$.
Soit, $\coordv{BD}{-17}{-33}$.\\%
\item À partir des coordonnées de B et $\vv{BD}$, on calcule ceux du point D.\\%
On a $\coord{B}{-2}{2}$ et $\coordv{BD}{-17}{-33}$. 
Donc $\coord{D}{-2  -  17}{2  -  33}$\\%

Et, finalement, $\coord{D}{-19}{-31}$%
\end{enumerate}%

\cor{Vecteurs et coordonnées.}%
\begin{enumerate}%
\item On calcule les coordonnées des vecteurs $\vv{AB}$ et $\vv{CA}$\\%
$\coordv{AB}{x_B - x_A}{y_B - y_A}$. Donc, $\coordv{AB}{-2 - (-3)}{1 - -1}$. 
Ainsi, $\coordv{AB}{1}{2}$\\%

$\coordv{CA}{x_A - x_C}{y_A - y_C}$. Donc, $\coordv{CA}{-3 - 3}{-1 - 5}$. 
Ainsi, $\coordv{CA}{-6}{-6}$\\%

\item On calcule maintenant, grâce à ces vecteurs, les coordonnées du vecteurs $\vv{BD}$\\%
$\vv{BD}=-5 \vv{AB}  +  4 \vv{CA}$.
Donc, $\coordv{BD}{-5 \times 1  +  4 \times (-6)}{
-5 \times 2  +  4 \times(-6)}$.
Soit, $\coordv{BD}{19}{14}$.\\%
\item À partir des coordonnées de B et $\vv{BD}$, on calcule ceux du point D.\\%
On a $\coord{B}{-2}{1}$ et $\coordv{BD}{19}{14}$. 
Donc $\coord{D}{-2  +  19}{1  +  14}$\\%

Et, finalement, $\coord{D}{17}{15}$%
\end{enumerate}%

\cor{Vecteurs et coordonnées.}%
\begin{enumerate}%
\item On calcule les coordonnées des vecteurs $\vv{AB}$ et $\vv{CA}$\\%
$\coordv{AB}{x_B - x_A}{y_B - y_A}$. Donc, $\coordv{AB}{-2 - (-5)}{5 - -3}$. 
Ainsi, $\coordv{AB}{3}{8}$\\%

$\coordv{CA}{x_A - x_C}{y_A - y_C}$. Donc, $\coordv{CA}{-5 - 4}{-3 - 2}$. 
Ainsi, $\coordv{CA}{-9}{-5}$\\%

\item On calcule maintenant, grâce à ces vecteurs, les coordonnées du vecteurs $\vv{BD}$\\%
$\vv{BD}=-2 \vv{AB}  +  2 \vv{CA}$.
Donc, $\coordv{BD}{-2 \times 3  +  2 \times (-9)}{
-2 \times 8  +  2 \times(-5)}$.
Soit, $\coordv{BD}{-24}{-26}$.\\%
\item À partir des coordonnées de B et $\vv{BD}$, on calcule ceux du point D.\\%
On a $\coord{B}{-2}{5}$ et $\coordv{BD}{-24}{-26}$. 
Donc $\coord{D}{-2  -  24}{5  -  26}$\\%

Et, finalement, $\coord{D}{-26}{-21}$%
\end{enumerate}%

\cor{Vecteurs et coordonnées.}%
\begin{enumerate}%
\item On calcule les coordonnées des vecteurs $\vv{AB}$ et $\vv{CA}$\\%
$\coordv{AB}{x_B - x_A}{y_B - y_A}$. Donc, $\coordv{AB}{-2 - (-3)}{2 - -2}$. 
Ainsi, $\coordv{AB}{1}{4}$\\%

$\coordv{CA}{x_A - x_C}{y_A - y_C}$. Donc, $\coordv{CA}{-3 - 2}{-2 - 5}$. 
Ainsi, $\coordv{CA}{-5}{-7}$\\%

\item On calcule maintenant, grâce à ces vecteurs, les coordonnées du vecteurs $\vv{BD}$\\%
$\vv{BD}=-5 \vv{AB}  +  3 \vv{CA}$.
Donc, $\coordv{BD}{-5 \times 1  +  3 \times (-5)}{
-5 \times 4  +  3 \times(-7)}$.
Soit, $\coordv{BD}{-20}{-41}$.\\%
\item À partir des coordonnées de B et $\vv{BD}$, on calcule ceux du point D.\\%
On a $\coord{B}{-2}{2}$ et $\coordv{BD}{-20}{-41}$. 
Donc $\coord{D}{-2  -  20}{2  -  41}$\\%

Et, finalement, $\coord{D}{-22}{-39}$%
\end{enumerate}%

\cor{Vecteurs et coordonnées.}%
\begin{enumerate}%
\item On calcule les coordonnées des vecteurs $\vv{AB}$ et $\vv{CA}$\\%
$\coordv{AB}{x_B - x_A}{y_B - y_A}$. Donc, $\coordv{AB}{-5 - (-3)}{3 - -3}$. 
Ainsi, $\coordv{AB}{-2}{6}$\\%

$\coordv{CA}{x_A - x_C}{y_A - y_C}$. Donc, $\coordv{CA}{-3 - 3}{-3 - 4}$. 
Ainsi, $\coordv{CA}{-6}{-7}$\\%

\item On calcule maintenant, grâce à ces vecteurs, les coordonnées du vecteurs $\vv{BD}$\\%
$\vv{BD}=4 \vv{AB}  +  2 \vv{CA}$.
Donc, $\coordv{BD}{4 \times -2  +  2 \times-6}{
4 \times 6  +  2 \times (-7)}$. 
Soit, $\coordv{BD}{-20}{10}$.\\%
\item À partir des coordonnées de B et $\vv{BD}$, on calcule ceux du point D.\\%
On a $\coord{B}{-5}{3}$ et $\coordv{BD}{-20}{10}$. 
Donc $\coord{D}{-5  -  20}{3  +  10}$\\%

Et, finalement, $\coord{D}{-25}{13}$%
\end{enumerate}%

\cor{Vecteurs et coordonnées.}%
\begin{enumerate}%
\item On calcule les coordonnées des vecteurs $\vv{AB}$ et $\vv{CA}$\\%
$\coordv{AB}{x_B - x_A}{y_B - y_A}$. Donc, $\coordv{AB}{-4 - (-4)}{2 - -2}$. 
Ainsi, $\coordv{AB}{0}{4}$\\%

$\coordv{CA}{x_A - x_C}{y_A - y_C}$. Donc, $\coordv{CA}{-4 - 5}{-2 - 3}$. 
Ainsi, $\coordv{CA}{-9}{-5}$\\%

\item On calcule maintenant, grâce à ces vecteurs, les coordonnées du vecteurs $\vv{BD}$\\%
$\vv{BD}=-4 \vv{AB}  +  3 \vv{CA}$.
Donc, $\coordv{BD}{-4 \times 0  +  3 \times-9}{
-4 \times 4  +  3 \times (-5)}$. 
Soit, $\coordv{BD}{-27}{-31}$.\\%
\item À partir des coordonnées de B et $\vv{BD}$, on calcule ceux du point D.\\%
On a $\coord{B}{-4}{2}$ et $\coordv{BD}{-27}{-31}$. 
Donc $\coord{D}{-4  -  27}{2  -  31}$\\%

Et, finalement, $\coord{D}{-31}{-29}$%
\end{enumerate}%

\cor{Vecteurs et coordonnées.}%
\begin{enumerate}%
\item On calcule les coordonnées des vecteurs $\vv{AB}$ et $\vv{CA}$\\%
$\coordv{AB}{x_B - x_A}{y_B - y_A}$. Donc, $\coordv{AB}{-3 - (-1)}{1 - -4}$. 
Ainsi, $\coordv{AB}{-2}{5}$\\%

$\coordv{CA}{x_A - x_C}{y_A - y_C}$. Donc, $\coordv{CA}{-1 - 1}{-4 - 5}$. 
Ainsi, $\coordv{CA}{-2}{-9}$\\%

\item On calcule maintenant, grâce à ces vecteurs, les coordonnées du vecteurs $\vv{BD}$\\%
$\vv{BD}=4 \vv{AB}  -  5 \vv{CA}$.
Donc, $\coordv{BD}{4 \times -2  -  5 \times-2}{
4 \times 5  -  5 \times (-9)}$. 
Soit, $\coordv{BD}{2}{65}$.\\%
\item À partir des coordonnées de B et $\vv{BD}$, on calcule ceux du point D.\\%
On a $\coord{B}{-3}{1}$ et $\coordv{BD}{2}{65}$. 
Donc $\coord{D}{-3  +  2}{1  +  65}$\\%

Et, finalement, $\coord{D}{-1}{66}$%
\end{enumerate}%

\cor{Vecteurs et coordonnées.}%
\begin{enumerate}%
\item On calcule les coordonnées des vecteurs $\vv{AB}$ et $\vv{CA}$\\%
$\coordv{AB}{x_B - x_A}{y_B - y_A}$. Donc, $\coordv{AB}{-3 - (-4)}{4 - -4}$. 
Ainsi, $\coordv{AB}{1}{8}$\\%

$\coordv{CA}{x_A - x_C}{y_A - y_C}$. Donc, $\coordv{CA}{-4 - 4}{-4 - 5}$. 
Ainsi, $\coordv{CA}{-8}{-9}$\\%

\item On calcule maintenant, grâce à ces vecteurs, les coordonnées du vecteurs $\vv{BD}$\\%
$\vv{BD}=5 \vv{AB}  +  2 \vv{CA}$.
Donc, $\coordv{BD}{5 \times 1  +  2 \times (-8)}{
5 \times 8  +  2 \times(-9)}$.
Soit, $\coordv{BD}{-11}{22}$.\\%
\item À partir des coordonnées de B et $\vv{BD}$, on calcule ceux du point D.\\%
On a $\coord{B}{-3}{4}$ et $\coordv{BD}{-11}{22}$. 
Donc $\coord{D}{-3  -  11}{4  +  22}$\\%

Et, finalement, $\coord{D}{-14}{26}$%
\end{enumerate}%

\cor{Vecteurs et coordonnées.}%
\begin{enumerate}%
\item On calcule les coordonnées des vecteurs $\vv{AB}$ et $\vv{CA}$\\%
$\coordv{AB}{x_B - x_A}{y_B - y_A}$. Donc, $\coordv{AB}{-5 - (-5)}{4 - -5}$. 
Ainsi, $\coordv{AB}{0}{9}$\\%

$\coordv{CA}{x_A - x_C}{y_A - y_C}$. Donc, $\coordv{CA}{-5 - 5}{-5 - 3}$. 
Ainsi, $\coordv{CA}{-10}{-8}$\\%

\item On calcule maintenant, grâce à ces vecteurs, les coordonnées du vecteurs $\vv{BD}$\\%
$\vv{BD}=4 \vv{AB}  +  5 \vv{CA}$.
Donc, $\coordv{BD}{4 \times 0  +  5 \times-10}{
4 \times 9  +  5 \times (-8)}$. 
Soit, $\coordv{BD}{-50}{-4}$.\\%
\item À partir des coordonnées de B et $\vv{BD}$, on calcule ceux du point D.\\%
On a $\coord{B}{-5}{4}$ et $\coordv{BD}{-50}{-4}$. 
Donc $\coord{D}{-5  -  50}{4  -  4}$\\%

Et, finalement, $\coord{D}{-55}{0}$%
\end{enumerate}%

\cor{Vecteurs et coordonnées.}%
\begin{enumerate}%
\item On calcule les coordonnées des vecteurs $\vv{AB}$ et $\vv{CA}$\\%
$\coordv{AB}{x_B - x_A}{y_B - y_A}$. Donc, $\coordv{AB}{-5 - (-3)}{2 - -2}$. 
Ainsi, $\coordv{AB}{-2}{4}$\\%

$\coordv{CA}{x_A - x_C}{y_A - y_C}$. Donc, $\coordv{CA}{-3 - 5}{-2 - 1}$. 
Ainsi, $\coordv{CA}{-8}{-3}$\\%

\item On calcule maintenant, grâce à ces vecteurs, les coordonnées du vecteurs $\vv{BD}$\\%
$\vv{BD}=4 \vv{AB}  +  3 \vv{CA}$.
Donc, $\coordv{BD}{4 \times -2  +  3 \times-8}{
4 \times 4  +  3 \times (-3)}$. 
Soit, $\coordv{BD}{-32}{7}$.\\%
\item À partir des coordonnées de B et $\vv{BD}$, on calcule ceux du point D.\\%
On a $\coord{B}{-5}{2}$ et $\coordv{BD}{-32}{7}$. 
Donc $\coord{D}{-5  -  32}{2  +  7}$\\%

Et, finalement, $\coord{D}{-37}{9}$%
\end{enumerate}%

\cor{Vecteurs et coordonnées.}%
\begin{enumerate}%
\item On calcule les coordonnées des vecteurs $\vv{AB}$ et $\vv{CA}$\\%
$\coordv{AB}{x_B - x_A}{y_B - y_A}$. Donc, $\coordv{AB}{-5 - (-4)}{5 - -3}$. 
Ainsi, $\coordv{AB}{-1}{8}$\\%

$\coordv{CA}{x_A - x_C}{y_A - y_C}$. Donc, $\coordv{CA}{-4 - 2}{-3 - 3}$. 
Ainsi, $\coordv{CA}{-6}{-6}$\\%

\item On calcule maintenant, grâce à ces vecteurs, les coordonnées du vecteurs $\vv{BD}$\\%
$\vv{BD}=2 \vv{AB}  -  2 \vv{CA}$.
Donc, $\coordv{BD}{2 \times -1  -  2 \times-6}{
2 \times 8  -  2 \times (-6)}$. 
Soit, $\coordv{BD}{10}{28}$.\\%
\item À partir des coordonnées de B et $\vv{BD}$, on calcule ceux du point D.\\%
On a $\coord{B}{-5}{5}$ et $\coordv{BD}{10}{28}$. 
Donc $\coord{D}{-5  +  10}{5  +  28}$\\%

Et, finalement, $\coord{D}{5}{33}$%
\end{enumerate}%

\cor{Vecteurs et coordonnées.}%
\begin{enumerate}%
\item On calcule les coordonnées des vecteurs $\vv{AB}$ et $\vv{CA}$\\%
$\coordv{AB}{x_B - x_A}{y_B - y_A}$. Donc, $\coordv{AB}{-4 - (-2)}{2 - -5}$. 
Ainsi, $\coordv{AB}{-2}{7}$\\%

$\coordv{CA}{x_A - x_C}{y_A - y_C}$. Donc, $\coordv{CA}{-2 - 3}{-5 - 4}$. 
Ainsi, $\coordv{CA}{-5}{-9}$\\%

\item On calcule maintenant, grâce à ces vecteurs, les coordonnées du vecteurs $\vv{BD}$\\%
$\vv{BD}=-5 \vv{AB}  -  5 \vv{CA}$.
Donc, $\coordv{BD}{-5 \times -2  -  5 \times-5}{
-5 \times 7  -  5 \times (-9)}$. 
Soit, $\coordv{BD}{35}{10}$.\\%
\item À partir des coordonnées de B et $\vv{BD}$, on calcule ceux du point D.\\%
On a $\coord{B}{-4}{2}$ et $\coordv{BD}{35}{10}$. 
Donc $\coord{D}{-4  +  35}{2  +  10}$\\%

Et, finalement, $\coord{D}{31}{12}$%
\end{enumerate}%

\cor{Vecteurs et coordonnées.}%
\begin{enumerate}%
\item On calcule les coordonnées des vecteurs $\vv{AB}$ et $\vv{CA}$\\%
$\coordv{AB}{x_B - x_A}{y_B - y_A}$. Donc, $\coordv{AB}{-3 - (-4)}{5 - -4}$. 
Ainsi, $\coordv{AB}{1}{9}$\\%

$\coordv{CA}{x_A - x_C}{y_A - y_C}$. Donc, $\coordv{CA}{-4 - 5}{-4 - 1}$. 
Ainsi, $\coordv{CA}{-9}{-5}$\\%

\item On calcule maintenant, grâce à ces vecteurs, les coordonnées du vecteurs $\vv{BD}$\\%
$\vv{BD}=3 \vv{AB}  +  4 \vv{CA}$.
Donc, $\coordv{BD}{3 \times 1  +  4 \times (-9)}{
3 \times 9  +  4 \times(-5)}$.
Soit, $\coordv{BD}{-33}{7}$.\\%
\item À partir des coordonnées de B et $\vv{BD}$, on calcule ceux du point D.\\%
On a $\coord{B}{-3}{5}$ et $\coordv{BD}{-33}{7}$. 
Donc $\coord{D}{-3  -  33}{5  +  7}$\\%

Et, finalement, $\coord{D}{-36}{12}$%
\end{enumerate}%

\cor{Vecteurs et coordonnées.}%
\begin{enumerate}%
\item On calcule les coordonnées des vecteurs $\vv{AB}$ et $\vv{CA}$\\%
$\coordv{AB}{x_B - x_A}{y_B - y_A}$. Donc, $\coordv{AB}{-4 - (-2)}{3 - -2}$. 
Ainsi, $\coordv{AB}{-2}{5}$\\%

$\coordv{CA}{x_A - x_C}{y_A - y_C}$. Donc, $\coordv{CA}{-2 - 4}{-2 - 5}$. 
Ainsi, $\coordv{CA}{-6}{-7}$\\%

\item On calcule maintenant, grâce à ces vecteurs, les coordonnées du vecteurs $\vv{BD}$\\%
$\vv{BD}=-4 \vv{AB}  +  4 \vv{CA}$.
Donc, $\coordv{BD}{-4 \times -2  +  4 \times-6}{
-4 \times 5  +  4 \times (-7)}$. 
Soit, $\coordv{BD}{-16}{-48}$.\\%
\item À partir des coordonnées de B et $\vv{BD}$, on calcule ceux du point D.\\%
On a $\coord{B}{-4}{3}$ et $\coordv{BD}{-16}{-48}$. 
Donc $\coord{D}{-4  -  16}{3  -  48}$\\%

Et, finalement, $\coord{D}{-20}{-45}$%
\end{enumerate}%

\cor{Vecteurs et coordonnées.}%
\begin{enumerate}%
\item On calcule les coordonnées des vecteurs $\vv{AB}$ et $\vv{CA}$\\%
$\coordv{AB}{x_B - x_A}{y_B - y_A}$. Donc, $\coordv{AB}{-1 - (-1)}{2 - -3}$. 
Ainsi, $\coordv{AB}{0}{5}$\\%

$\coordv{CA}{x_A - x_C}{y_A - y_C}$. Donc, $\coordv{CA}{-1 - 5}{-3 - 5}$. 
Ainsi, $\coordv{CA}{-6}{-8}$\\%

\item On calcule maintenant, grâce à ces vecteurs, les coordonnées du vecteurs $\vv{BD}$\\%
$\vv{BD}=-3 \vv{AB}  -  5 \vv{CA}$.
Donc, $\coordv{BD}{-3 \times 0  -  5 \times-6}{
-3 \times 5  -  5 \times (-8)}$. 
Soit, $\coordv{BD}{30}{25}$.\\%
\item À partir des coordonnées de B et $\vv{BD}$, on calcule ceux du point D.\\%
On a $\coord{B}{-1}{2}$ et $\coordv{BD}{30}{25}$. 
Donc $\coord{D}{-1  +  30}{2  +  25}$\\%

Et, finalement, $\coord{D}{29}{27}$%
\end{enumerate}%

\cor{Vecteurs et coordonnées.}%
\begin{enumerate}%
\item On calcule les coordonnées des vecteurs $\vv{AB}$ et $\vv{CA}$\\%
$\coordv{AB}{x_B - x_A}{y_B - y_A}$. Donc, $\coordv{AB}{-5 - (-5)}{5 - -2}$. 
Ainsi, $\coordv{AB}{0}{7}$\\%

$\coordv{CA}{x_A - x_C}{y_A - y_C}$. Donc, $\coordv{CA}{-5 - 3}{-2 - 1}$. 
Ainsi, $\coordv{CA}{-8}{-3}$\\%

\item On calcule maintenant, grâce à ces vecteurs, les coordonnées du vecteurs $\vv{BD}$\\%
$\vv{BD}=5 \vv{AB}  +  4 \vv{CA}$.
Donc, $\coordv{BD}{5 \times 0  +  4 \times-8}{
5 \times 7  +  4 \times (-3)}$. 
Soit, $\coordv{BD}{-32}{23}$.\\%
\item À partir des coordonnées de B et $\vv{BD}$, on calcule ceux du point D.\\%
On a $\coord{B}{-5}{5}$ et $\coordv{BD}{-32}{23}$. 
Donc $\coord{D}{-5  -  32}{5  +  23}$\\%

Et, finalement, $\coord{D}{-37}{28}$%
\end{enumerate}%

\cor{Vecteurs et coordonnées.}%
\begin{enumerate}%
\item On calcule les coordonnées des vecteurs $\vv{AB}$ et $\vv{CA}$\\%
$\coordv{AB}{x_B - x_A}{y_B - y_A}$. Donc, $\coordv{AB}{-2 - (-4)}{1 - -3}$. 
Ainsi, $\coordv{AB}{2}{4}$\\%

$\coordv{CA}{x_A - x_C}{y_A - y_C}$. Donc, $\coordv{CA}{-4 - 4}{-3 - 3}$. 
Ainsi, $\coordv{CA}{-8}{-6}$\\%

\item On calcule maintenant, grâce à ces vecteurs, les coordonnées du vecteurs $\vv{BD}$\\%
$\vv{BD}=-3 \vv{AB}  -  4 \vv{CA}$.
Donc, $\coordv{BD}{-3 \times 2  -  4 \times (-8)}{
-3 \times 4  -  4 \times(-6)}$.
Soit, $\coordv{BD}{26}{12}$.\\%
\item À partir des coordonnées de B et $\vv{BD}$, on calcule ceux du point D.\\%
On a $\coord{B}{-2}{1}$ et $\coordv{BD}{26}{12}$. 
Donc $\coord{D}{-2  +  26}{1  +  12}$\\%

Et, finalement, $\coord{D}{24}{13}$%
\end{enumerate}%

\end{document}